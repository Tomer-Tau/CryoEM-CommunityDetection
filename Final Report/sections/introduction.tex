\center

\section{Introduction}

\raggedright

Single-particle reconstruction in \acrshort{cryo-EM} is a powerful image-processing tool used to determine the 3D structure of biological macromolecular complexes.
2D images (micrographs) of a macromolecule are taken by an electron microscope, and essentially the set of all micrographs for a given macromolecule spans a 3D model of the macromolecule. Thus, single-particle reconstruction is using the micrographs to build a 3D model of the macromolecule.\\

Due to high sensitivity of the biological macromolecules to radiation damage, electron microscope provides limited electron doses when producing micrographs. This and the low contrast of micrographs result in \acrshort{cryo-EM} data having very low \acrfull{SNR}\cite{9016106}.\\

\acrshort{cryo-EM} technology has the potential to offer the ability to analyze different functional and conformational states of macromolecules, an important ability for the field of molecular biology. Practically, it entails the classification of heterogeneous \acrshort{cryo-EM} data.\\

Many different approaches for \acrshort{cryo-EM} data classification have been developed. Typically likelihood optimization algorithms and Bayesian inference frameworks are used to deal with data heterogeneity\cite{sigworth1998maximum,scheres2005fast,scheres2014beam,song2013flexibility,chowdhury2015structural}. In our project we will use \acrfull{CD} from the field of complex networks by converting \acrshort{cryo-EM} data into a graph and applying \acrshort{CD} on it to obtain classification of the heterogeneous \acrshort{cryo-EM} data.



For the sake of an abstraction of the \acrshort{cryo-EM} data we will use the Heterogeneous \acrfull{MRA} statistical model. In our project we use the simplified  1D version of the model.


\subsection{Some subsection}
\lipsum
